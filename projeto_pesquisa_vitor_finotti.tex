%% abtex2-modelo-projeto-pesquisa.tex, v<VERSION> laurocesar
%% Copyright 2012-<COPYRIGHT_YEAR> by abnTeX2 group at http://www.abntex.net.br/
%%
%% This work may be distributed and/or modified under the
%% conditions of the LaTeX Project Public License, either version 1.3
%% of this license or (at your option) any later version.
%% The latest version of this license is in
%%   http://www.latex-project.org/lppl.txt
%% and version 1.3 or later is part of all distributions of LaTeX
%% version 2005/12/01 or later.
%%
%% This work has the LPPL maintenance status `maintained'.
%%
%% The Current Maintainer of this work is the abnTeX2 team, led
%% by Lauro César Araujo. Further information are available on
%% http://www.abntex.net.br/
%%
%% This work consists of the files abntex2-modelo-projeto-pesquisa.tex
%% and abntex2-modelo-references.bib
%%

% ------------------------------------------------------------------------
% ------------------------------------------------------------------------
% abnTeX2: Modelo de Projeto de pesquisa em conformidade com
% ABNT NBR 15287:2011 Informação e documentação - Projeto de pesquisa -
% Apresentação
% ------------------------------------------------------------------------
% ------------------------------------------------------------------------

\documentclass[
	% -- opções da classe memoir --
	12pt,				% tamanho da fonte
	%openright,			% capítulos começam em pág ímpar (insere página vazia caso preciso)
	oneside,			% para impressão em recto e verso. Oposto a oneside
	a4paper,			% tamanho do papel.
	% -- opções da classe abntex2 --
	%chapter=TITLE,		% títulos de capítulos convertidos em letras maiúsculas
	%section=TITLE,		% títulos de seções convertidos em letras maiúsculas
	%subsection=TITLE,	% títulos de subseções convertidos em letras maiúsculas
	%subsubsection=TITLE,% títulos de subsubseções convertidos em letras maiúsculas
	% -- opções do pacote babel --
	english,			% idioma adicional para hifenização
	french,				% idioma adicional para hifenização
	spanish,			% idioma adicional para hifenização
	brazil,				% o último idioma é o principal do documento
	]{abntex2}

% ---
% PACOTES
% ---

% ---
% Pacotes fundamentais
% ---
\usepackage{lmodern}			% Usa a fonte Latin Modern
\usepackage[T1]{fontenc}		% Selecao de codigos de fonte.
\usepackage[utf8]{inputenc}		% Codificacao do documento (conversão automática dos acentos)
\usepackage{indentfirst}		% Indenta o primeiro parágrafo de cada seção.
\usepackage{color}				% Controle das cores
\usepackage{graphicx}			% Inclusão de gráficos
\usepackage{microtype} 			% para melhorias de justificação
% ---

% ---
% Pacotes adicionais
% ---
\usepackage{pdflscape}
\usepackage{adjustbox}
\usepackage{capa-epusp-abntex2}
%\usepackage{xcolor}
%\color{blue}
% ---

% ---
% Pacotes de citações
% ---
\usepackage[alf]{abntex2cite}	% Citações padrão ABNT

% ---
% Informações de dados para CAPA e FOLHA DE ROSTO
% ---
\titulo{Identificação de plantas daninhas por redes neurais em FPGAs}
\autor{Vitor Finotti Ferreira}
\local{São Paulo}
\data{2017 - $3^{o}$ Período}
\instituicao{%
  Universidade de São Paulo
  \par
  Escola Politécnica
  \par
  Programa de Pós-Graduação - Engenharia de Computação e Sistemas Digitais}
\tipotrabalho{Dissetação (Mestrado)}
\areaconcentracao{Engenharia Elétrica - Engenharia da Computação}
% O preambulo deve conter o tipo do trabalho, o objetivo,
% o nome da instituição e a área de concentração
\preambulo{Proposta de dissertação de mestrado sobre a implementação de redes neurais no contexto de Field Programable Gate Arrays para identificação de plantas daninhas.}
% ---

% ---
% Configurações de aparência do PDF final

% alterando o aspecto da cor azul
\definecolor{blue}{RGB}{41,5,195}

% informações do PDF
\makeatletter
\hypersetup{
     	%pagebackref=true,
		pdftitle={\@title},
		pdfauthor={\@author},
    	pdfsubject={\imprimirpreambulo},
	    pdfcreator={LaTeX with abnTeX2},
		pdfkeywords={abnt}{latex}{abntex}{abntex2}{projeto de pesquisa},
		colorlinks=true,       		% false: boxed links; true: colored links
    	linkcolor=blue,          	% color of internal links
    	citecolor=blue,        		% color of links to bibliography
    	filecolor=magenta,      		% color of file links
		urlcolor=blue,
		bookmarksdepth=4
}
\makeatother
% ---

% ---
% Espaçamentos entre linhas e parágrafos
% ---

% O tamanho do parágrafo é dado por:
\setlength{\parindent}{1.3cm}

% Controle do espaçamento entre um parágrafo e outro:
\setlength{\parskip}{0.2cm}  % tente também \onelineskip

% ---
% compila o indice
% ---
\makeindex
% ---

% ----
% Início do documento
% ----
\begin{document}

% Seleciona o idioma do documento (conforme pacotes do babel)
%\selectlanguage{english}
\selectlanguage{brazil}

% Retira espaço extra obsoleto entre as frases.
\frenchspacing

% ----------------------------------------------------------
% ELEMENTOS PRÉ-TEXTUAIS
% ----------------------------------------------------------
% \pretextual

% ---
% Capa
% ---
\imprimircapa
% ---

% ---
% Folha de rosto
% ---
\imprimirfolhaderosto
% ---

% ---
% NOTA DA ABNT NBR 15287:2011, p. 4:
%  ``Se exigido pela entidade, apresentar os dados curriculares do autor em
%     folha ou página distinta após a folha de rosto.''
% ---

% ---
% RESUMOS
% ---

% resumo em português
\setlength{\absparsep}{18pt} % ajusta o espaçamento dos parágrafos do resumo
\begin{resumo}
 O cultivo de cana-de-açúcar tem um papel de destaque dentro do agronegócio brasileiro, tendo participação tanto como matéria prima alimentícia quanto energética. Quando o foco do cultivo é a produção de produtos orgânicos, limitações quanto aos pesticidas e agrotóxicos usados geram uma necessidade de encontrar alternativas para controlar pragas tais como o cipó \textit{Ipomoea grandifolia}. Este trabalho propõe implementar uma rede neural em FPGA capaz de identificar esta planta daninha por imagem, de forma a ser a primeira etapa em um sistema de combate a ela.

 \textbf{Palavras-chave}: Cana-de-açúcar, FPGA, Redes Neurais.
\end{resumo}

% resumo em inglês
\begin{resumo}[Abstract]
 \begin{otherlanguage*}{english} % deixar para atualizar por ultimo
  The sugarcane cultivation has a prominent role in the Brazilian agribusiness, being used not only in the food sector but also as an energy resource. In situations where the cultivation focus is the growth and production of organic products, limitations on pesticides and agrochemicals usage leads to the urge to find alternatives to control pests, such as \textit{[quote the vine here]}. This work proposes to implement a neural network in FPGA capable of identifying this weed, working as a first stage of a system to combat it.

   \vspace{\onelineskip}
 
   \noindent 
   \textbf{Keywords}: Sugarcane, FPGA, Neural Networks.
 \end{otherlanguage*}
\end{resumo}


% ---
% inserir o sumario
% ---
%\pdfbookmark[0]{\contentsname}{toc}
%\tableofcontents*
%\cleardoublepage
% ---


% ----------------------------------------------------------
% ELEMENTOS TEXTUAIS
% ----------------------------------------------------------
\textual

% ----------------------------------------------------------
% Introdução
% ----------------------------------------------------------

\chapter*[Introdução]{Introdução}

  O agronegócio tem uma expressiva participação na economia brasileira, fomentando não apenas a agricultura e a pecuária em si, mas toda uma infraestrutura tecnológia e logística que dá suporte ao setor. Isto engloba diversas áreas correlatas à agropecuária, tanto a montante como a jusante, envolvendo: produção de insumo para a agropecuária, produção de matérias-primas agropecuárias, processamento dessas matérias-primas, distribuição e demais serviços, até o consumo final ou exportação~\cite{cepea}. No ano de 2016, o agronegócio representou 23\% do PIB nacional, movimentando R\$ 1,425 trilhões~\cite{cna}, sendo um fundamental fator atenuante na crise econômica que o país vivencia~\cite{}.
  
  Dentre as diversas atividades que compõe o agronegócio, o cultivo de cana-de-açúcar tem um papel importante, sendo que apenas sua  cadeia de produção foi responsável por movimentar R\$ 164,181 bilhões em 2016 \cite{cna}. Em meio às necessidades da produção da cana-de-açúcar está um nicho focado em produtos orgânicos, cuja produção é caracterizada pela sustentabilidade social, ambiental e econômica~\cite{organicos_canal_rural}. Dentre suas particularidades, encontra-se a não utilização de agrotóxicos, hormônios, drogas veterinárias, adubos químicos, antibióticos ou transgênicos em qualquer fase da produção. Os benefícios adquiridos por estas restrições contrastam com a dificuldade em aumentar a escala da produção e manter o custo competitivo, o que concentra este mercado em pequenos produtores, limitando o acesso de seus benefícios à população. Embora com grande potencial, o mercado de produtos orgânicos ainda é caracterizado por uma tímida oferta em comparação com a crescente demanda \cite{organicos_carta_capital}.
  
  O desenvolvimento de tecnologias que possam dar suporte ao crescimento deste setor se mostram fundamentais para desenvolver o seu potencial nacional, cuja participação no mercado é 10 vezes menor à de mercados tradicionais, como os EUA ou Alemanha \cite{organicos_carta_capital}. 
  
% Ligar com o texto dos objetivos explicando como é feito o controle atualmente em lavouras organicas.
% Hoje, o produtor contrata um ser humano pra ir de tempos em tempos lá e cortar manualmente. O que acontece é que a lavoura de cana de acucar é agressiva para o ser humano, oferencendo riscos como cortes, fogo, animais peconhentos, etc. A ideia final é identificar remotamente os cipos para que o corte possa ser feito somente quando necessario, preferencialmente por um equipamento autonomo.
% Nas lavouras organicas nao pode-se entrar com maquinario pesado pois o solo é "vivo", com uma microfauna importante para manutençao da lavoura. As maquinas pesadas danificam este ambiente e são consideradas inadequadas para o cultivo organico. Todo maquinario, emsmo leve, é preparado para não compactar o solo.
% Robos agricolas não sao adequados para supervisao remota pois dependem de bateria e a comunicacao é precaria por causa da presenca de agua (nas canas de acucar).
    
\chapter*[Objetivos]{Objetivos}
 Este trabalho tem como objetivo implementar uma rede neural convolucional em FPGA capaz de identificar o \textit{[citar o cipó aqui]}, erva daninha que atinge a cana-de-açúcar. Tal identificação seria o primeiro estágio de um sistema para o controle desta praga, dificuldade que presentes em produções orgânicas que não podem utilizar agrotóxicos para isso. A implementação em FPGA se mostra atraente tanto por sua alta capacidade de paralelização quanto pelo consumo energético reduzido em comparação com outras arquiteturas de hardware, como GPUs. Tais características se mostram importantes no contexto de uma plantação rural, onde o acesso a energia e a internet são limitados.

\chapter*[Revisão Bibliográfica]{Revisão Bibliográfica}

\section*{Plantas daninhas no cultivo de cana-de-açúcar}
\textit{[Esperar mais informações]}

\section*{Redes Neurais Convolucionais}
Dentre as diversas categorias de redes neurais em estudo, as redes neurais convolucionais têm se destacado em campos como o de reconhecimento de imagem.

\section*{Field-Programmable Gate Arrays}


\chapter*[Contextualização e Justificativa]{Contextualização e Justificativa}
O desenvolvimento no campo de aprendizado de máquina nos últimos anos
fez com que estes algoritmos se popularizassem, sendo aplicados nas mais
diversas áreas do conhecimento e indústria. Utilizando-se do treinamento
feito com conjunto adequado de amostras, aplicados em um modelo
apropriado, é possível obter resultados iguais ou superiores aos
adquiridos por análise humana a um mesmo problema.

Existem situações, contudo, nas quais as próprias características do
problema exigem que um hardware específico seja usado no treinamento ou
na inferência de um algoritmo, tanto para melhorar o desempenho da rede
quanto para satisfazer um critério de latência de processamento. Neste
contexto, os Field Programmable Gate Arrays (FPGAs) podem oferecer uma
plataforma interessante por conta de sua reconfigurabilidade, grande
I/O, paralelismo e consumo energético inferior à tecnologias
concorrentes.

Este projeto se propõe a investigar a aplicação destes algoritmos em
FPGAs, analisando as características em termos de desempenho, buscando
também situações que possam se beneficiar especialmente pelo uso delas.



\chapter*[Metodologia]{Metodologia}
A metodologia de estudo consistirá em:
\begin{enumerate}
    \item Aprofundar os estudos das teorias de análise estatística,
      algoritmos de aprendizado de máquina e desenvolvimento de hardware em
      FPGAs
    \item Analisar as características do \textit{[Citar cipó aqui]} e implementar uma rede neural capaz de fazer seu reconhecimento de uma imagem sua
    \item Implementar a rede encontrada em uma placa FPGA, com uso real ou emulação
      de entradas
    \item Validar a implementação através de simulações e testes empíricos
\end{enumerate}
	
\chapter*[Recursos Utilizados]{Recursos Utilizados}
A proposta é desenvolver o projeto utilizando ferramentas \textit{open
 source} para diminuir o custo de implementação e facilitar a
disseminação da tecnologia criada sempre que possível.
Para a fase de desenvolvimento, um computador com sistema operacional
Linux e \textit{softwares} de desenvolvimento de sistemas como Python,
R, VHDL, Emacs entre outros. Dado que muitas vezes é necessário usar
\textit{softwares} específicos para FPGAs de um fabricante, pode ser
necessário usar softwares privados para a simulação e desenvolvimento.
Almeja-se usar \textit{evaluation boards} de FPGAs para fazer proveito
de toda a infraestrutura e recursos disponibilizados por elas, se
possível com modelos atuais. Planeja-se procurar diversas bibliotecas
\textit{open source} para reutilização de código, tanto para as etapas
de estudo de algoritmos de aprendizado de máquina quanto para a
implementação dos mesmos em FPGA. Também pretende-se utilizar a
infraestrutura dos laboratórios da Escola Politécnica e para
desenvolvimento e validação do trabalho feito. Os danos para teste e validação dos experimentos feitos serão obtidos \textit{[especificar fonte dos dados obtidos, provavelmente da empresa interessada no estudo]}.
        
\chapter*[Cronograma Previsto]{Cronograma Previsto}
\begin{enumerate}
	\item Disciplina 1; %item 1
	\item Disciplina 2; %item 2
	\item Disciplina 3; %item 3
	\item Disciplina 4; %item 4
	\item Disciplina 5; %item 5
	\item Pesquisa bibliográfica; %item 6
	\item Análise da aplicação; %item 7
	\item Implementação do sistema; %item 8
	\item Validação do sistema; %item 9
	\item Exame de qualificação; %item 10
	\item Escrita da tese; %item 11
	\item Defesa da tese; %item 12
	\item Revisão/comentários; %item 13
	\item Preparação de artigos e participação em congressos; %item 14
\end{enumerate}

\begin{landscape}
	\centering
\begin{adjustbox}{width=\textheight, totalheight=\textwidth,keepaspectratio}
%\begin{table}
%	\caption{Cronograma de desenvolvimento do projeto}\label{cronograma}
%	\vspace{.3cm}
\begin{tabular}{|c|c|c|c|c|c|c|c|c|c|c|c|c|c|c|c|c|c|c|c|c|c|c|c|c|}
	\hline
	\raisebox{0pt}[12pt][6pt]  & \multicolumn{4}{|c|}{2017} & \multicolumn{12}{|c|}{2018} & \multicolumn{8}{|c|}{2019} \\
	\hline
	\raisebox{0pt}[12pt][6pt]	        &S&O&N&D&J&F&M&A&M&J&J&A&S&O&N&D&J&F&M&A&M&J&J&A\\
	\hline
	\raisebox{0pt}[12pt][6pt]{Item 1}	&X&X&X&X& & & & & & & & & & & & & & & & & & & & \\
	\hline
	\raisebox{0pt}[12pt][6pt]{Item 2}   &X&X&X&X& & & & & & & & & & & & & & & & & & & & \\
	\hline
	\raisebox{0pt}[12pt][6pt]{Item 3}	& & & & &X&X&X&X&X& & & & & & & & & & & & & & & \\
	\hline
	\raisebox{0pt}[12pt][6pt]{Item 4}   & & & & & & & & & &X&X&X& & & & & & & & & & & & \\
	\hline
	\raisebox{0pt}[12pt][6pt]{Item 5}   & & & & & & & & & &X&X&X& & & & & & & & & & & & \\
	\hline
	\raisebox{0pt}[12pt][6pt]{Item 6}	& & & & &X&X&X& & & & & & & & & & & & & & & & & \\
	\hline
	\raisebox{0pt}[12pt][6pt]{Item 7}	& & & & & & & & & &X&X&X& & & & & & & & & & & & \\
	\hline
	\raisebox{0pt}[12pt][6pt]{Item 8}   & & & & & & & & & & & & &X&X&X&X&X&X&X& & & & & \\
	\hline
	\raisebox{0pt}[12pt][6pt]{Item 9}   & & & & & & & & & & & & & &X&X&X&X&X&X& & & & & \\
	\hline
	\raisebox{0pt}[12pt][6pt]{Item 10}  & & & & & & & & & & &X&X& & & & & & & & & & & & \\
	\hline
	\raisebox{0pt}[12pt][6pt]{Item 11}	& & & & & & & & & & & & & & & & &X&X&X&X& & & & \\
	\hline
	\raisebox{0pt}[12pt][6pt]{Item 12}  & & & & & & & & & & & & & & & & & & & &X&X& & & \\
	\hline
	\raisebox{0pt}[12pt][6pt]{Item 13}  & & & & & & & & & & & & & & & & & & & & & &X&X& \\
	\hline
	\raisebox{0pt}[12pt][6pt]{Item 14}  & & & & & & & & & & & & & & & & & & & &X&X&X&X&X\\
	\hline
\end{tabular}
%\end{table}
\end{adjustbox}
\end{landscape}


% ----------------------------------------------------------
% ELEMENTOS PÓS-TEXTUAIS
% ----------------------------------------------------------
\postextual

% ----------------------------------------------------------
% Referências bibliográficas
% ----------------------------------------------------------

\bibliography{biblio}
\nocite{farabet2011large,
        friedman2009elements,
        james2013introduction,
        ovtcharov2015accelerating,
        russell2003modern}

% ---

\end{document}

%%% Local Variables:
%%% mode: latex
%%% TeX-master: t
%%% End:
